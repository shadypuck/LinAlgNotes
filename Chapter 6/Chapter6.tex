\documentclass{article}

\usepackage[letterpaper,margin=1in]{geometry}
\usepackage{fancyhdr}
\usepackage{marginnote}
\usepackage{amsmath}
\usepackage{amssymb}
\usepackage{tikz}
\usepackage{csquotes}

\MakeOuterQuote{"}

\pagestyle{fancy}
\fancyhf{}
\rfoot{Labalme \thepage}
\rhead{(H) Linear Algebra}

\reversemarginpar

\begin{document}




\lhead{Chapter 6: Eigenvalues}
\section*{Introduction to Eigenvalues and Eigenvectors}
\begin{itemize}
    \item \marginnote{1/28:}[-8.5mm]$Ax=b=\lambda x$
    \item $Ax=\lambda x$, $\lambda\in\mathbb{F}$, $x\in\mathbb{R}^n$
    \item $\lambda$ is an eigenvalue. $\lambda x$ is an eigenvector.
    \begin{equation*}
        A=
        \begin{bmatrix}
            3 & 1\\
            1 & 3\\
        \end{bmatrix}
    \end{equation*}
    \item $
        x =
        \begin{bmatrix}
            1\\
            1\\
        \end{bmatrix}
    $ is an eigenvector of $A$ with corresponding eigenvalue of $4$.
    \item $
        \begin{bmatrix}
            3 & 1\\
            1 & 3\\
        \end{bmatrix}
        \begin{bmatrix}
            1\\
            1\\
        \end{bmatrix}
        =
        \begin{bmatrix}
            4\\
            4\\
        \end{bmatrix}
        = 4
        \begin{bmatrix}
            1\\
            1\\
        \end{bmatrix}
    $
    \begin{center}
        \begin{tikzpicture}
            \draw (-2,0) -- (2,0);
            \draw (0,-2) -- (0,2);
            \draw [thick,->] (0,0) -- node[right]{$x$} (0.5,0.5);
            \draw [thick,dashed,->] (0,0) -- node[right]{$Ax$} (2,2);
        \end{tikzpicture}
    \end{center}
    \begin{align*}
        Ax &= \lambda x\\
        Ax-\lambda x &= 0\\
        Ax-\lambda Ix &= 0\\
        (A-\lambda I)x &= 0
    \end{align*}
    \item $(A-\lambda I)x=0 \Rightarrow x\in N(A-\lambda I)^[$\footnote{To have a null space, $A-\lambda I$ has free columns.}$^] \Rightarrow |A-\lambda I|=0$
    \item $
        \begin{bmatrix}
            3 & 1\\
            1 & 3\\
        \end{bmatrix}
        -
        \begin{bmatrix}
            \lambda & 0\\
            0 & \lambda\\
        \end{bmatrix}
        =
        \begin{bmatrix}
            3-\lambda & 1\\
            1 & 3-\lambda\\
        \end{bmatrix}
    $
    \item $
        \begin{vmatrix}
            3-\lambda & 1\\
            1 & 3-\lambda\\
        \end{vmatrix}
        = 0
    $
    \begin{align*}
        0 &= (3-\lambda)^2-1^2\\
        &= 3^2-6\lambda+\lambda^2-1\\
        &= \lambda^2-6\lambda+8\\
        &= (\lambda-4)(\lambda-2)
    \end{align*}
    \item $\lambda=4,2$.
    \item $\lambda^2-6\lambda+8$ is the \textbf{characteristic polynomial} of $A$.
    \item $
        A-2I =
        \begin{bmatrix}
            -1 & 1\\
            1 & -1\\
        \end{bmatrix}
        ,\ 
        x =
        \begin{bmatrix}
            1\\
            -1\\
        \end{bmatrix}
        \in N(A-2I)
    $.
    \item $
        A-4I =
        \begin{bmatrix}
            -1 & 1\\
            1 & -1\\
        \end{bmatrix}
        ,\ 
        x =
        \begin{bmatrix}
            1\\
            1\\
        \end{bmatrix}
        \in N(A-4I)
    $.
    \item "Eigenspace" is not $\mathbb{R}^2$, but two lines in $\mathbb{R}^2$, specifically $y=\pm x$.
    \begin{itemize}
        \item $y=\pm x$ comes from $c_1\begin{bmatrix}1\\1\end{bmatrix}$ and $c_2\begin{bmatrix}1\\-1\end{bmatrix}$.
    \end{itemize}

    %%%%%%%%%%%%%%%%%%%%%%

    \marginnote{1/29:}\begin{equation*}
        A =
        \begin{bmatrix}
            2 & -2 & 3\\
            0 & 3 & -2\\
            0 & -1 & 2\\
        \end{bmatrix}
    \end{equation*}
    \begin{align*}
        P(\lambda) &= |A-\lambda I|\\
        &=
        \begin{vmatrix}
            2-\lambda & -2 & 3\\
            0 & 3-\lambda & -2\\
            0 & -1 & 2-\lambda\\
        \end{vmatrix}\\
        &= -1
        \begin{vmatrix}
            2-\lambda & 3\\
            0 & -2\\
        \end{vmatrix}
        (-1)^{3+2} + (2-\lambda)
        \begin{vmatrix}
            2-\lambda & -2\\
            0 & 3-\lambda\\
        \end{vmatrix}
        (-1)^{3+3}\\
        &= ((2-\lambda)(-2))+(2-\lambda)((2-\lambda)(3-\lambda))\\
        &= -4+2\lambda+(2-\lambda)^2(3-\lambda)\\
        &= -4+2\lambda+(4-4\lambda+\lambda^2)(3-\lambda)\\
        &= -4+2\lambda+12-4\lambda-12\lambda+4\lambda^2+3\lambda^2-\lambda^3\\
        &= -\lambda^3+7\lambda^2-14\lambda+8\\
        &= -(\lambda-1)(\lambda-2)(\lambda-4)
    \end{align*}
    \begin{align*}
        A-I &=
        \begin{bmatrix}
            1 & -2 & 3\\
            0 & 2 & -2\\
            0 & -1 & 1\\
        \end{bmatrix}&
        A-2I &=
        \begin{bmatrix}
            0 & -2 & 3\\
            0 & 1 & -2\\
            0 & -1 & 0\\
        \end{bmatrix}&
        A-4I &=
        \begin{bmatrix}
            -2 & -2 & 3\\
            0 & -1 & -2\\
            0 & -1 & -2\\
        \end{bmatrix}&
    \end{align*}
    \item $P(\lambda)$ is positive when $n\in 2\mathbb{N}$, negative otherwise.
    \begin{itemize}
        \item Signs flip term to term (think about binomial expansion).
    \end{itemize}
    \item Coefficients of the $n-1$ degree term is the sum of the diagonal entries.
    \item Coefficient of the $0^\text{th}$ degree term is $|A|$.
    \begin{itemize}
        \item $P_\lambda(0) = |A-0\cdot I| = |A|$.
    \end{itemize}
    \item Product of the eigenvalues is $|A|$.
    \begin{itemize}
        \item Think about expanding the factorization.
    \end{itemize}
    \item Eigenvalues of $U$ are the diagonal values.
    \begin{itemize}
        \item $\lambda_1\lambda_2\cdots\lambda_n=|A|$, which is the product of the diagonal entries.
        \item $\lambda_1+\cdots+\lambda_n=\text{trace}(A)$, which is the sum of the diagonal entries.
    \end{itemize}
    \item $Ax=\lambda x$
    \begin{itemize}
        \item $A^2x=AAx=A\lambda x=\lambda Ax=\lambda\lambda x=\lambda^2x$
    \end{itemize}
\end{itemize}



\section*{Similarity}
\begin{itemize}
    \item \marginnote{1/30:}$A\sim B$$^[$\footnote{$A$ "is similar to" $B$}$^]$ iff $\exists\ S:A=SBS^{-1},\ B=S^{-1}AS$.
    \begin{enumerate}
        \item If $A\sim B$, then $|A|=|B|$.
        \begin{align*}
            B &= S^{-1}AS\\
            |B| &= |S^{-1}AS|\\
            |B| &= |S^{-1}||A||S|\\
            |B| &= \frac{1}{|S|}|A||S|\\
            |B| &= |A|
        \end{align*}
        \item If $A\sim B$, then they share the same characteristic polynomial.
        \begin{align*}
            B &= S^{-1}AS\\
            |B-\lambda I| &= |S^{-1}AS-\lambda I|\\
            &= |S^{-1}AS-\lambda S^{-1}IS|\\
            &= |S^{-1}S(A-\lambda I)|\\
            &= |I(A-\lambda I)|\\
            |B-\lambda I| &= |A-\lambda I|
        \end{align*}
        \begin{itemize}
            \item If they have the same characteristic polynomial, $\therefore$ $A$ and $B$ have the same eigenvalues.
        \end{itemize}
    \end{enumerate}
    \item What is the best possible $B$ if $A\sim B$?
    \begin{itemize}
        \item Sparse.
        \item Diagonal.
        \item $
            A=[\text{ugly}]\quad\rightarrow\quad
            B=\begin{bmatrix}
                \lambda_1 &  & 0\\
                 & \ddots & \\
                0 &  & \lambda_n\\
            \end{bmatrix}
            =\Lambda
        $
    \end{itemize}
    \item \textbf{Diagonalization}:
    \begin{align*}
        A &= S\Lambda S^{-1}\\
        \Lambda &= SAS^{-1}
    \end{align*}
    \item $A=S\Lambda S^{-1}$
    \begin{itemize}
        \item $A^2 = AA = S\Lambda S^{-1}S\Lambda S^{-1} = S\Lambda\Lambda S^{-1} = S\Lambda^2 S^{-1}$
        \item $A^k = S\Lambda^k S^{-1}$
        \item $
            A^k = S
            \begin{bmatrix}
                {\lambda_1}^k &  & 0\\
                 & \ddots & \\
                0 &  & {\lambda_n}^k\\
            \end{bmatrix}
            S^{-1}
        $
    \end{itemize}
    \item Diagonalize the following matrix $A$.
    \begin{equation*}
        A =
        \begin{bmatrix}
            -1 & 0 & 1\\
            3 & 0 & -3\\
            1 & 0 & -1\\
        \end{bmatrix}
    \end{equation*}
    \begin{itemize}
        \item Find the characteristic polynomial.
        \begin{align*}
            |A-\lambda I| &=
            \begin{vmatrix}
                -1-\lambda & 0 & 1\\
                3 & 0-\lambda & -3\\
                1 & 0 & -1-\lambda\\
            \end{vmatrix}\\
            &= (-1-\lambda)((-\lambda)(-1-\lambda))+(-1)(-\lambda)\\
            &= -\lambda(-1-\lambda)^2+\lambda\\
            &= -\lambda(1+2\lambda+\lambda^2)+\lambda\\
            &= -\lambda^3-2\lambda^2\\
            &= -\lambda^2(\lambda+2)
        \end{align*}
        \item Find the eigenvalues: $\lambda_1=\lambda_2=0$, $\lambda_3=-2$
        \item \textbf{Algebraic multiplicity} of $\lambda_1,\lambda_2$ is 2.
        \item A.M. of $\lambda_3$ is 1.
        \item $
            A-0I =
            \begin{bmatrix}
                -1 & 0 & 1\\
                3 & 0 & -3\\
                1 & 0 & -1\\
            \end{bmatrix}
        $
        \item $\text{rank}(A-0I)=1 \Rightarrow \dim(N(A-0I))=2$
        \item The 2 directly above is the \textbf{geometric multiplicity}.
        \item $A$ is diagonalizable iff A.M. of $\lambda_i=\text{G.M.}$
        
        %%%%%%%%%%%%%%%%%%%%%%%%%
        
        \item \marginnote{1/31:}Eigenvectors are $
            x_1=
            \begin{bmatrix}
                1\\
                0\\
                1\\
            \end{bmatrix}
        $ and $
            x_2=
            \begin{bmatrix}
                0\\
                1\\
                0\\
            \end{bmatrix}
        $.
        \item $
        A+2I =
        \begin{bmatrix}
            1 & 0 & 1\\
            3 & 2 & -3\\
            1 & 0 & 1\\
        \end{bmatrix}
    $
        \item Eigenvector is $
            x_3=
            \begin{bmatrix}
                1\\
                -3\\
                -1\\
            \end{bmatrix}
        $
        \item Use an $S$ matrix of eigenvectors.
        \item $
            A = S\Lambda S^{-1} = \frac{1}{2}
            \begin{bmatrix}
                0 & 1 & 1\\
                1 & 0 & -3\\
                0 & 1 & -1\\
            \end{bmatrix}
            \begin{bmatrix}
                0 &  & \\
                 & 0 & \\
                 &  & -2\\
            \end{bmatrix}
            \begin{bmatrix}
                3 & 2 & -3\\
                1 & 0 & 1\\
                1 & 0 & -1\\
            \end{bmatrix}
        $
        \item Note that $
            A^{9752} = \frac{1}{2}
            \begin{bmatrix}
                0 & 1 & 1\\
                1 & 0 & -3\\
                0 & 1 & -1\\
            \end{bmatrix}
            \begin{bmatrix}
                0 &  & \\
                 & 0 & \\
                 &  & (-2)^{9752}\\
            \end{bmatrix}
            \begin{bmatrix}
                3 & 2 & -3\\
                1 & 0 & 1\\
                1 & 0 & -1\\
            \end{bmatrix}
        $
    \end{itemize}
    \item \textbf{Algebraic multiplicity}: The number of repeated roots to a polynomial. For all of the roots, it adds up to $n$ ($n$-square matrix). \emph{Also known as} \textbf{A.M.}
    \item \textbf{Geometric multiplicity}: The number of eigenvectors produced from each root. For all of the roots, it may not add up to $n$ ($n$-square matrix). $\dim(N(A-\lambda I))$. \emph{Also known as} \textbf{G.M.}
    \item A nondiagonalizable example:
    \begin{equation*}
        A =
        \begin{bmatrix}
            1 & 1 & 0\\
            0 & 1 & 1\\
            0 & 0 & 4\\
        \end{bmatrix}
    \end{equation*}
    \begin{itemize}
        \item $\lambda_1 = \lambda_2 = 1$ and $\lambda_3 = 4$.
        \item $\lambda_1$ and $\lambda_2$ have $\text{A.M.}={\color{red}2}$.
        \item $\lambda_3$ has $\text{A.M.}=1$.
        \item $
            A-I=
            \begin{bmatrix}
                0 & 1 & 0\\
                0 & 0 & 1\\
                0 & 0 & 3\\
            \end{bmatrix}
        $
        \item $\text{rank}(A-I)=2\Rightarrow\dim(N(A-I))=1\Rightarrow\text{G.M.}={\color{red}1}$.
        \item $
            x_1=
            \begin{bmatrix}
                1\\
                0\\
                0\\
            \end{bmatrix}
        $
        \item $
            A-4I=
            \begin{bmatrix}
                -3 & 1 & 0\\
                0 & -3 & 1\\
                0 & 0 & 0\\
            \end{bmatrix}
        $
        \item $
            x_2=
            \begin{bmatrix}
                1\\
                3\\
                9\\
            \end{bmatrix}
        $
        \item $S$ would be $3\times 2$ and, thus, not square, so $\nexists\ S^{-1}$$^[$\footnote{At a later date, we will look at an analogy of projections to diagonalization that finds the "best possible" diagonalization (which may not be perfectly diagonal).}$^]$.
    \end{itemize}
    \item \textbf{Canonical} (form): An accepted way of expressing something.
\end{itemize}




\end{document}