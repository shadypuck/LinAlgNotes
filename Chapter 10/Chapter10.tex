\documentclass{article}

\usepackage[letterpaper,margin=1in]{geometry}
\usepackage{fancyhdr}
\usepackage{marginnote}
\usepackage{amsmath}
\usepackage{amssymb}
\usepackage{tikz}
\usepackage{csquotes}
\usepackage{tabularx}
\usepackage[hidelinks]{hyperref}

\MakeOuterQuote{"}

\colorlet{grx}{green!50!black}

\newcommand{\R}{\mathbb{R}}
\newcommand{\T}{\text{T}}
\newcommand{\dy}{\text{d}y}
\newcommand{\dx}{\text{d}x}
\newcommand{\dd}[2][]{\frac{\text{d}#1}{\text{d}#2}}
\newcommand{\e}{\text{e}}

\newenvironment{tchart}[3]
{
    \renewcommand{\arraystretch}{#1}
    \tabularx{\linewidth}{X|X}
    \multicolumn{1}{c|}{\textbf{#2}} & \multicolumn{1}{c}{\textbf{#3}}\\
    \hline
}{
    \endtabularx
    \renewcommand{\arraystretch}{1}
}

\newenvironment{amatrix}[1]{
    \left[\begin{array}{@{}*{#1}{c}|c@{}}
}{
    \end{array}\right]
}

\renewcommand{\labelitemiii}{\scriptsize$\blacksquare$}

\pagestyle{fancy}
\fancyhf{}
\rfoot{Labalme \thepage}
\rhead{(H) Linear Algebra}

\reversemarginpar

\begin{document}




\lhead{Chapter 10: Complex Vectors and Matrices}
\section*{Complex Linear Independence: Decomplexification}
\begin{itemize}
    \item \marginnote{4/7:}When given a complex system of equations, it is necessary to \textbf{decomplexify} it.
    \item \textbf{Decomplexify}: To model a complex system of equations with a strictly real system for the purpose of applying the tenets of real linear algebra to it.
    \item Consider the following complex system of equations.
    \begin{align*}
        (2+i)x_1+(1+i)x_2 &= 3+6i\\
        (3-i)x_1+(2-2i)x_2 &= 7-i
    \end{align*}
    \begin{itemize}
        \item The solutions will be complex numbers: $x_1=a_1+ib_1$ and $x_2=a_2+ib_2$, where $a_1,a_2,b_1,b_2\in\R$.
    \end{itemize}
    \item Transform it into a matrix system of equations. Separate the real and complex parts, and factor out all instances of the imaginary number $i$ so that it is a coefficient to any complex matrix.
    \begin{align*}
        \begin{bmatrix}
            2+i & 1+i\\
            3-i & 2-2i\\
        \end{bmatrix}
        \begin{bmatrix}
            a_1+ib_1\\
            a_2+ib_2\\
        \end{bmatrix}
        &=
        \begin{bmatrix}
            3+6i\\
            7-i\\
        \end{bmatrix}\\
        \left(
            \begin{bmatrix}
                2 & 1\\
                3 & 2\\
            \end{bmatrix}
            +
            \begin{bmatrix}
                i & i\\
                -i & -2i\\
            \end{bmatrix}
        \right)\left(
            \begin{bmatrix}
                a_1\\
                a_2\\
            \end{bmatrix}
            +
            \begin{bmatrix}
                ib_1\\
                ib_2\\
            \end{bmatrix}
        \right) &= \left(
            \begin{bmatrix}
                3\\
                7\\
            \end{bmatrix}
            +
            \begin{bmatrix}
                6i\\
                -i\\
            \end{bmatrix}
        \right)\\
        \underbrace{\left(
            \begin{bmatrix}
                2 & 1\\
                3 & 2\\
            \end{bmatrix}
            +i
            \begin{bmatrix}
                1 & 1\\
                -1 & -2\\
            \end{bmatrix}
        \right)}_A \underbrace{\left(
            \begin{bmatrix}
                a_1\\
                a_2\\
            \end{bmatrix}
            +i
            \begin{bmatrix}
                b_1\\
                b_2\\
            \end{bmatrix}
        \right)}_x &= \underbrace{\left(
            \begin{bmatrix}
                3\\
                7\\
            \end{bmatrix}
            +i
            \begin{bmatrix}
                6\\
                -1\\
            \end{bmatrix}
        \right)}_b
    \end{align*}
    \item Foil the left side of the above equation\footnote{Note that the minus sign appears in the real component because, when multiplying the two "last" parts, $i^2=-1$.Note that the minus sign appears in the real component because, when multiplying the two "last" parts, $i^2=-1$.}.
    \begin{equation*}
        \left(
            \begin{bmatrix}
                2 & 1\\
                3 & 2\\
            \end{bmatrix}
            \begin{bmatrix}
                a_1\\
                a_2\\
            \end{bmatrix}
            -
            \begin{bmatrix}
                1 & 1\\
                -1 & -2\\
            \end{bmatrix}
            \begin{bmatrix}
                b_1\\
                b_2\\
            \end{bmatrix}
        \right)+i\left(
            \begin{bmatrix}
                2 & 1\\
                3 & 2\\
            \end{bmatrix}
            \begin{bmatrix}
                b_1\\
                b_2\\
            \end{bmatrix}
            +
            \begin{bmatrix}
                1 & 1\\
                -1 & -2\\
            \end{bmatrix}
            \begin{bmatrix}
                a_1\\
                a_2\\
            \end{bmatrix}
        \right) =
        \begin{bmatrix}
            3\\
            7\\
        \end{bmatrix}
        +i
        \begin{bmatrix}
            6\\
            -1\\
        \end{bmatrix}
    \end{equation*}
    \item Split the above system of equations into a real system of equations and a complex system of equations by setting equal to each other the real components of each side and the imaginary components of each side.
    \begin{align*}
        \begin{bmatrix}
            2 & 1\\
            3 & 2\\
        \end{bmatrix}
        \begin{bmatrix}
            a_1\\
            a_2\\
        \end{bmatrix}
        -
        \begin{bmatrix}
            1 & 1\\
            -1 & -2\\
        \end{bmatrix}
        \begin{bmatrix}
            b_1\\
            b_2\\
        \end{bmatrix}
        &=
        \begin{bmatrix}
            3\\
            7\\
        \end{bmatrix}\\
        \begin{bmatrix}
            2 & 1\\
            3 & 2\\
        \end{bmatrix}
        \begin{bmatrix}
            b_1\\
            b_2\\
        \end{bmatrix}
        +
        \begin{bmatrix}
            1 & 1\\
            -1 & -2\\
        \end{bmatrix}
        \begin{bmatrix}
            a_1\\
            a_2\\
        \end{bmatrix}
        &=
        \begin{bmatrix}
            6\\
            -1\\
        \end{bmatrix}
    \end{align*}
    \item Multiply out the matrices above to yield a system of four equations.
    \begin{align*}
        2a_1+a_2-b_1-b_2 &= 3\\
        3a_1+2a_2+b_1+2b_2 &= 7\\
        a_1+a_2+2b_1+b_2 &= 6\\
        -a_1-2a_2+3B_1+2B_2 &= -1
    \end{align*}
    \item Condense the above system of equations into a single matrix system of equations.
    \begin{align*}
        \begin{bmatrix}
            2 & 1 & -1 & -1\\
            3 & 2 & 1 & 2\\
            1 & 1 & 2 & 1\\
            -1 & -2 & 3 & 2\\
        \end{bmatrix}
        \begin{bmatrix}
            a_1\\
            a_2\\
            b_1\\
            b_2\\
        \end{bmatrix}
        =
        \begin{bmatrix}
            3\\
            7\\
            6\\
            -1\\
        \end{bmatrix}
    \end{align*}
    \item Solve for $a_1$, $a_2$, $b_1$, and $b_2$ using an augmented matrix and Gauss-Jordan elimination.
    \begin{equation*}
        \begin{amatrix}{4}
            2 & 1 & -1 & -1 & 3\\
            3 & 2 & 1 & 2 & 7\\
            1 & 1 & 2 & 1 & 6\\
            -1 & -2 & 3 & 2 & -1\\
        \end{amatrix}
        \rightarrow
        \begin{amatrix}{4}
            1 & 0 & 0 & 0 & 1\\
            0 & 1 & 0 & 0 & 2\\
            0 & 0 & 1 & 0 & 2\\
            0 & 0 & 0 & 1 & -1\\
        \end{amatrix}
    \end{equation*}
    \begin{equation*}
        \begin{bmatrix}
            a_1\\
            a_2\\
            b_1\\
            b_2\\
        \end{bmatrix}
        =
        \begin{bmatrix}
            1\\
            2\\
            2\\
            -1\\
        \end{bmatrix}
    \end{equation*}
    \item From these four values, the original solutions $x_1=a_1+ib_1$ and $x_2=a_2+ib_2$ can be found.
    \begin{align*}
        x_1 &= 1+2i\\
        x_2 &= 2-i
    \end{align*}
\end{itemize}




\end{document}